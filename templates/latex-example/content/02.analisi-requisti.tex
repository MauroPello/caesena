\section{Analisi dei requisiti}
La registrazione alla piattaforma Rius.Co. sarà obbligatoria per poter accedere al Marketplace, gli utenti dovranno inserire la propria e-mail identificativa, il nome dell'account, la password, un'immagine di profilo e la città in cui vivono. La piattaforma inizialmente sarà accessibile esclusivamente tramite il sito web, verrà successivamente introdotta un'applicazione mobile della quale è già presente il prototipo \href{https://mauro886267.invisionapp.com/console/share/7Z10U19EHJ/476334736}{visualizzabile qui} \cite{Prototipo}.  
\medskip

\begin{figure}[hb]
    \hfill
    \subfigure[Acquista]{\includegraphics[scale=.11]{images/ACQUISTA.png}}
    \hfill
    \subfigure[Offri]{\includegraphics[scale=.11]{images/OFFRI.png}}
    \hfill
    \subfigure[Transazioni]{\includegraphics[scale=.11]{images/TRANSAZIONI.png}}
    \hfill
    \caption{Prototipo dell'applicazione mobile.}
\end{figure}

Gli utenti potranno ottenere i Green Coin necessari per commerciare sul Marketplace attraverso le seguenti modalità:
\begin{itemize}
    \item \textbf{Registrazione sulla piattaforma}: verrà fornito un Green Coin alla registrazione di un nuovo account;
    \item \textbf{Condivisione della piattaforma sui propri canali Social}: gli utenti che sceglieranno di condividere e pubblicizzare online l'applicazione otterranno un Green Coin. Questa modalità sarà rinnovabile mensilmente, perciò gli utenti potranno ottenere un Green Coin ogni mese semplicemente pubblicizzando la piattaforma che ne guadagnerà in popolarità ed utenti, fidelizzando anche i propri clienti; 
    \item \textbf{Offrire prodotti nel Marketplace}: l'utente potrà inserire prodotti a patto che siano funzionanti e in buone condizioni, otterrà un Green Coin per ogni prodotto inserito nel Marketplace e ceduto ad un altro utente;
    \item \textbf{Dimostrare di essere in una situazione di difficoltà economica} attraverso la dichiarazione dei redditi o l'attestazione ISEE, a tali utenti verranno forniti mensilmente dai 2 ai 5 Green Coin.
\end{itemize}
Il Green Coin è stato pensato per risolvere una problematica importante comune a tutti gli attuali centri del riuso fisici: la presenza di utenti che abusano della piattaforma richiedendo una grande mole di prodotti, senza mai contribuire al sistema offrendone altrettanti. 
\medskip

Gli utenti dopo essersi registrati potranno accedere alla piattaforma ed usufruire dei servizi che essa offre, sarà quindi possibile:  
\begin{itemize}
    \item \textbf{Acquistare un prodotto di cui si necessita attraverso la pagina “Acquista”.} Nella pagina saranno presenti tutti i prodotti offerti da utenti membri della piattaforma nella vicinanza dell'utente. Verrà data anche la possibilità di cercare un determinato prodotto all'interno del Marketplace. L'utente interessato ad un prodotto potrà spendere un Green Coin per contattare l'utente offerente e accordarsi per lo scambio. Nel caso in cui lo scambio non si sia concluso con successo l'utente richiedente otterrà indietro il Green Coin che aveva speso;
    \item \textbf{Offrire un prodotto attraverso la pagina “Offri”.} Nella pagina sarà possibile inserire un prodotto da offrire nel Marketplace, saranno richieste una breve descrizione e delle foto rappresentative del prodotto, ne verrà resa pubblica anche la posizione. Nel caso un utente sia interessato al prodotto il venditore verrà contattato da quest'ultimo e otterrà un Green Coin nel caso lo scambio si concluda con successo;
    \item \textbf{Controllare le proprie transazioni attraverso la pagina “Transazioni”.} L'utente potrà visualizzare tutte le sue transazioni, i dettagli relativi ad esse e lo stato attuale: “In Corso”, “Conclusa senza Successo” e “Conclusa con Successo”. Saranno anche visualizzabili i movimenti di Green Coin in entrata ed uscita;
    \item \textbf{Visualizzare e modificare il proprio profilo attraverso la pagina “Profilo”.} L'utente potrà visualizzare il proprio profilo e modificare i dati relativi ad esso mantenendo aggiornate informazioni importanti come la città di residenza.
\end{itemize}