\section{Sviluppo}
\subsection{Testing automatizzato}
Per effettuare i test necessari per verificare la correttezza delle funzionalità del nostro software abbiamo utilizzato la suite dedicata JUnit 5. I test che siamo andati ad implementare sono:
\begin{itemize}
    \item GameSetTest: questa classe permette di testare la corretta creazione di vari GameSet (esempio: citySet, FieldSet), l'unione di due GameSet e il piazzamento di meeple all'interno dei GameSet. Sono presenti test anche per il controllo della corretta assegnazione e modifica dei punti relativi ai GameSet, è inclusa anche la verifica della gestione della corretta chiusura di un GameSet e dell'assegnazione dei punti ai Player aventi Meeple sul GameSet appena chiuso.
    \item NormalMeepleTest: questa classe permette di testare i metodi appartenenti al NormalMeeple e il posizionamento di esso all'interno di una Tile.
    \item PlayerTest: questa classe ci permette di testare i metodi appartenenti al Player e la modifica progressiva dei punti a lui assegnati.
    \item TileTest: questa classe ci permette di testare la corretta creazione di Tile e i metodi ad essa associata, più in specifico la chiusura di una TileSection e il suo scorrimento circolare all'interno della Tile.
    \item ToStringBuilderTest: questa classe ci permette di testare la classe ToStringBuilder, in particolar modo l'ottenimento di una stringa rappresentativa per un qualsiasi Object. Infine verifica la consistenza delle rappresentazioni letterali degli oggetti, garantendo che rimangano invariate.
    \item DirectionTest: questa classe ci permette di testare l'enum Direction e il suo metodo match che controlla che la differenza delle coordinate di due punti combaci con la direzione fornita.
    \item GameSetTileMediatorTest: questa classe ci permette di testare i vari metodi della classe GameSetTileMediator, per esempio la rotazione di una Tile con i GameSet ad essa associata, e la corretta corrispondenza tra i GameSet e le Tile con le proprie TileSection.
\end{itemize}

Per quanto riguarda la parte di view non sono stati scritti dei veri e propri test JUnit, ma si è ritenuto più opportuno testare manualmente le funzionalità principali:
\begin{itemize}
    \item utilizzando schermi di diverse dimensioni e risoluzioni per verificare che il gioco funzionasse ugualmente per ognuno di essi e che il ridimensionamento fosse corretto.
    \item utilizzando variabili di debug per accedere direttamente ad una Scene da testare evitando tutte le altre Scene che secondo la logica di gioco sarebbero dovute essere state visualizzate prima.
\end{itemize}

\subsection{Metodologia di lavoro}
Ci siamo focalizzati molto sulla fase di progettazione in modo da avere interfacce ben definite una volta iniziata la fase di sviluppo. Sempre in questa fase, una parte che ha richiesto altrettanta attenzione è stata la ricerca di tecniche per effettuare una trasposizione informatica fedele al gioco da tavolo originale.
\medskip

Inizialmente abbiamo suddiviso il lavoro in modo che ognuno potesse lavorare parallelamente. Ci sono stati però dei casi in cui è stato deciso, per efficienza e semplicità, che un componente del gruppo lavorasse su classi non di sua competenza, questo è avvenuto per due possibili ragioni:
\begin{itemize}
    \item Individuazione e risoluzione tempestiva e continuativa di bug o errori di sorta.
    \item Implentazioni di tecniche o pattern da applicare nel medesimo modo in più classi e interfacce.
\end{itemize}

Per quanto riguarda il DVCS abbiamo deciso di utilizzare Git, in particolare abbiamo creato due diversi branch, uno per la stesura della relazione e l'altro per lo sviluppo del codice. Ogni componente ha quindi avuto la resposabilità di clonare localmente la repository e di fare le operazioni di \textit{pull}, \textit{push} ed eventualmente \textit{merge}.

\subsubsection*{Mauro Pellonara}
In autonomia mi sono occupato di:
\begin{itemize}
    \item Interfaccia grafica principale (GUI in it.unibo.caesena.view)
    \item Menù principale e di pausa (StartScene e PauseScene in it.unibo.caesena.view.scene)
    \item Vari componenti con scopi generici (in it.unibo.caesena.view.components.common)
    \item Componenti relativi al player (PlayerInput e PlayerImage in it.unibo.caesena.view.components.player)
    \item Ridimensionamento dinamico di FooterComponentImpl (in it.unibo.caesena.view.components)
    \item "Caching" delle immagini (RemainingMeeplesComponent in it.unibo.caesena.view.components.meeple)
    \item Traduzione in italiano del gioco (LocaleHelper in it.unibo.caesena.view)
    \item Gestione di inizio e reset della partita (Controller in it.unibo.caesena.controller)
    \item Controllo della chiusura dei GameSet (Controller in it.unibo.caesena.controller)
    \item Gestione delle UserInterfaces (Controller in it.unibo.caesena.controller)
    \item Creazione di Tile (TileFactory e TileBuilder in it.unibo.caesena.model.tile)
    \item GameSetTileMediator con il relativo test (in it.unibo.caesena.model)
    \item Meeple con il relativo test (in it.unibo.caesena.model.meeple)
\end{itemize}
In collaborazione mi sono occupato di:
\begin{itemize}
    \item Gestione della fine della partita con Alessandro Martini (Controller in it.unibo.caesena.controller)
    \item Tile e il relativo test con Samuele Giancarli (in it.unibo.caesena.model.tile)
    \item TileType con Davide Speziali (in it.unibo.caesena.model.tile)
\end{itemize}

\subsubsection*{Alessandro Martini}
In autonomia mi sono occupato di:
\begin{itemize}
    \item Implementazione e gestione dei vari GameSet (package it.unibo.caesena.model.gameset)
    \item Implementazione della GameOverScene (package it.unibo.caesena.view.scene)
    \item Implementazione BasicComponent (package it.unibo.caesena.view;)
    \item Implementazione UpdatableComponent (package it.unibo.caesena.view;)
    \item Implementazione Configuration Loader (package it.unibo.caesena.controller)
    \item Controller.endTurn() (package it.unibo.caesena.controller)
    \item Controller.discardCurrentTile (package it.unibo.caesena.controller)
    \item Controller.isCurrentTilePlaceable (package it.unibo.caesena.controller)
    \item Controller.getEmptyNeighbouringPositions (package it.unibo.caesena.controller)
    \item Controller.isPositionOccupied (package it.unibo.caesena.controller)
\end{itemize}
In collaborazione mi sono occupato di:
\begin{itemize}
    \item Gestione della fine della partita con Mauro Pellonara (Controller in it.unibo.caesena.controller)
\end{itemize}

\subsubsection*{Davide Speziali}
In autonomia mi sono occupato di:
\begin{itemize}
    \item
\end{itemize}
In collaborazione mi sono occupato di:
\begin{itemize}
    \item
\end{itemize}

\subsubsection*{Samuele Giancarli}
In autonomia mi sono occupato di:
\begin{itemize}
    \item Gestione della rotazione e del piazzamento tessere (package it.unibo.caesena.controller)
    \item Inserimento delle Tessere pennant in TileType (it.unibo.caesena.model.tile)
    \item Implementazione delle Factory per le tessere PENNANT in TileFactoryWithBuilder (it.unibo.caesena.model.tile)
    \item Implementazione del FooterComponent (package it.unibo.caesena.view.components;)
    \item Implementazione del SideBarComponent (package it.unibo.caesena.view.components)
    \item Implementazione del LeaderBoardComponent (package it.unibo.caesena.view.components.player)
    \item Implementazione del RemainingMeeplesComponent (package it.unibo.caesena.view.components.meeple)
    \item Implementazione del ResourceUtil (it.unibo.caesena.utils)
    \item Implementazione di TileSection (package it.unibo.caesena.model.tile)
\end{itemize}

In collaborazione mi sono occupato di:
\begin{itemize}
    \item Sviluppo dell'intero package Tile e relativi test con Mauro Pellonara
\end{itemize}

\subsection{Note di sviluppo}
\subsubsection*{Mauro Pellonara}
\begin{itemize}
    \item Utilizzo di ResourceBundle e Locale per supporto di più lingue: \href{https://github.com/MauroPello/OOP22-caesena/blob/5fae6d5fd9f79fe417edee91115e3c74f08d6e0d/src/main/java/it/unibo/caesena/view/LocaleHelper.java}{permalink} dell'esempio
    \item Utilizzo di tipi generici: \href{https://github.com/MauroPello/OOP22-caesena/blob/5fae6d5fd9f79fe417edee91115e3c74f08d6e0d/src/main/java/it/unibo/caesena/view/scene/Scene.java}{permalink} dell'esempio
    \item Utilizzo di Stream, Lambda e Method reference, alcuni esempi sono:
    \begin{itemize}
        \item \href{https://github.com/MauroPello/OOP22-caesena/blob/b817cad200379d563deda711ebf7773b572fb061/src/main/java/it/unibo/caesena/model/GameSetTileMediatorImpl.java#L107}{Permalink} del 1° esempio
        \item \href{https://github.com/MauroPello/OOP22-caesena/blob/dff66c49fc9fb8e6bdd8d328986f4b9ee3a4b2dc/src/main/java/it/unibo/caesena/controller/ControllerImpl.java#L134}{Permalink} del 2° esempio
        \item \href{https://github.com/MauroPello/OOP22-caesena/blob/dff66c49fc9fb8e6bdd8d328986f4b9ee3a4b2dc/src/main/java/it/unibo/caesena/controller/ControllerImpl.java#L390}{Permalink} del 3° esempio
        \item \href{https://github.com/MauroPello/OOP22-caesena/blob/b817cad200379d563deda711ebf7773b572fb061/src/main/java/it/unibo/caesena/model/GameSetTileMediatorImpl.java#L203}{Permalink} del 4° esempio
        \item \href{https://github.com/MauroPello/OOP22-caesena/blob/b817cad200379d563deda711ebf7773b572fb061/src/main/java/it/unibo/caesena/model/GameSetTileMediatorImpl.java#L213}{Permalink} del 5° esempio
    \end{itemize}
\end{itemize}

\subsubsection*{Alessandro Martini}
\begin{itemize}
    \item Utilizzo di Stream: sono presenti all'interno delle classi sviluppati porzioni di codice contenente il costrutto funzionale Stream: \url{https://github.com/MauroPello/OOP22-caesena/blob/2e7e1877928505c25e791d2e1ba6e5ebffffca64/src/main/java/it/unibo/caesena/model/gameset/GameSetImpl.java}
    \item Uso di JSON Parser: utilizzo di un JSON parser per prendere tutte le tile presenti nel gioco (memorizzate nel file config.json) \url{https://github.com/MauroPello/OOP22-caesena/blob/2e7e1877928505c25e791d2e1ba6e5ebffffca64/src/main/java/it/unibo/caesena/controller/ConfigurationLoader.java}
\end{itemize}

\subsubsection*{Davide Speziali}

\subsubsection*{Samuele Giancarli}
\begin{itemize}
    \item \href{https://github.com/MauroPello/OOP22-caesena/blob/main/src/main/java/it/unibo/caesena/view/components/meeple/RemainingMeeplesComponentImpl.java#L64}{Permalink}
    \item \href{https://github.com/MauroPello/OOP22-caesena/blob/e35a57ad5bd0b2fce619acdc5060cb8dc1d2d1e9/src/main/java/it/unibo/caesena/model/GameSetTileMediatorImpl.java#L123}{Permalink}
\end{itemize}
