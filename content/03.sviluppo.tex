\section{Sviluppo}
\subsection{Testing automatizzato}
Per effettuare i test necessari per i corretti controlli di funzionalità del nostro software abbiamo utilizzato la suite dedicata JUnit 5. I test che siamo andati ad implementare riguardando
\begin{itemize}
    \item ControllerTest: questa classe permette di testare la costruzione delle Tiles (che non ritornino) errori (.isEmpty()) durante il loro posizionamento, l'aggiunta di un nuovo giocatore e il ritorno del corretto numero di meeple's a inizio partita
    \item GameSetTest: questa classe permette di testare la corretta creazione di vari GameSet (esempio: citySet, FieldSet), l'unione di due GameSet e il testing sui relativi meeple presenti sulle tile dei due GameSet appena uniti 
    \item NormalMeepleTest: questa classe permette di testare il corretto numero di meeple presenti e se sono stati piazzati in maniera corretta
    \item PlayerTest: questa classe ci permette di testare il get di tutte le informazioni relative al giocatore, e la corretta gestione dello score relativo al giocatore
    \item TileTest: questa classe ci permettedi andar a testare la corretta creazione e posizionamento di tiles, il get delle informazioni relative a quella tile (.isPlaced()), la corretta gestione della rotazione di una tile, la corretta gestione della chiusura di una section (chiusura di un'immagine) e lo shift della TileSection.
    \item ToStringBuilderTest:
    Non sono stati creati test JUnit nella View poichè 
\end{itemize}

\subsection{Metodologia di lavoro}
Nelle prime ore dedicate per il nostro progetto, abbiamo deciso e impostato l'architettura software da noi usata, l'MVC, successivamente abbiamo impostato e sviluppato le interfacce generiche che poi sarebbero state utilizzate come guida nello sviluppo software nelle singole parti di model, controller e view. Abbiamo utilizzato il software visto a lezione GitHub per la gestione e condivisione di software da noi creato e sviluppato individualmente.
\subsubsection*{Mauro Pellonara} 
In autonomia ho sviluppato:
\begin{itemize}
    \item 
\end{itemize}
In collaborazione ho sviluppato:
\begin{itemize}
    \item 
\end{itemize}

\subsubsection*{Alessandro Martini}
In autonomia ho sviluppato:
\begin{itemize}
    \item 
\end{itemize}
In collaborazione ho sviluppato:
\begin{itemize}
    \item 
\end{itemize}

\subsubsection*{Davide Speziali}
In autonomia ho sviluppato:
\begin{itemize}
    \item 
\end{itemize}
In collaborazione ho sviluppato:
\begin{itemize}
    \item 
\end{itemize}

\subsubsection*{Samuele Giancarli}
In autonomia ho sviluppato:
\begin{itemize}
    \item 
\end{itemize}
In collaborazione ho sviluppato:
\begin{itemize}
    \item 
\end{itemize}

\subsection{Note di sviluppo}
\subsubsection*{Mauro Pellonara} 

\subsubsection*{Alessandro Martini}

\subsubsection*{Davide Speziali}

\subsubsection*{Samuele Giancarli}
