\section{Sviluppo}
\subsection{Testing automatizzato}
Per effettuare i test necessari per i corretti controlli di funzionalità del nostro software abbiamo utilizzato la suite dedicata JUnit 5. I test che siamo andati ad implementare riguardando
\begin{itemize}
    \item ControllerTest: questa classe permette di testare la costruzione delle Tiles (che non ritornino) errori (.isEmpty()) durante il loro posizionamento, l'aggiunta di un nuovo giocatore e il ritorno del corretto numero di meeple's a inizio partita
    \item GameSetTest: questa classe permette di testare la corretta creazione di vari GameSet (esempio: citySet, FieldSet), l'unione di due GameSet, il testing sui relativi meeple presenti sulle tile dei due GameSet appena uniti. Sono presenti test anche per il controllo della corretta assegnazione e modifica dei punti relativi ai GameSet, inclusa anche la gestione della corretta chiusura di un GameSet e dell'assegnazione dei punti ai player aventi meeples sul GameSet appena chiuso. 
    \item NormalMeepleTest: questa classe permette di testare il corretto numero di meeple presenti e se sono stati piazzati in maniera corretta
    \item PlayerTest: questa classe ci permette di testare il get di tutte le informazioni relative al giocatore, e la corretta gestione dello score relativo al giocatore
    \item TileTest: questa classe ci permettedi andar a testare la corretta creazione e posizionamento di tiles, il get delle informazioni relative a quella tile (.isPlaced()), la corretta gestione della rotazione di una tile, la corretta gestione della chiusura di una section (chiusura di un'immagine) e lo shift della TileSection.
    \item ToStringBuilderTest:
\end{itemize}
Non sono stati creati test JUnit nella View poichè 

\subsection{Metodologia di lavoro}
Nelle prime ore dedicate per il nostro progetto, abbiamo deciso e impostato l'architettura software da noi usata, l'MVC, successivamente abbiamo impostato e sviluppato le interfacce generiche che poi sarebbero state utilizzate come guida nello sviluppo software nelle singole parti di model, controller e view. Abbiamo utilizzato il software visto a lezione GitHub per la gestione e condivisione di software da noi creato e sviluppato individualmente.
\subsubsection*{Mauro Pellonara} 
In autonomia ho sviluppato:
\begin{itemize}
    \item 
\end{itemize}
In collaborazione ho sviluppato:
\begin{itemize}
    \item 
\end{itemize}

\subsubsection*{Alessandro Martini}
In autonomia ho sviluppato:
\begin{itemize}
    \item Implementazione e gestione dei vari GameSet (package it.unibo.caesena.model.gameset)
    \item Implementazione della GameOverView (package it.unibo.caesena.view)
    \item Implementazione Configuration Loader (package it.unibo.caesena.controller) 
    \item Controller.endTurn() (package it.unibo.caesena.controller)
    \item Controller.discardCurrentTile (package it.unibo.caesena.controller)
    \item Controller.isCurrentTilePlaceable (package it.unibo.caesena.controller)
    \item Controller.getEmptyNeighbouringPositions (package it.unibo.caesena.controller)
    \item Controller.isPositionOccupied (package it.unibo.caesena.controller)
\end{itemize}
In collaborazione ho sviluppato:
\begin{itemize}
    \item con Mauro Pellonara: 
    \begin{itemize}
        \item Controller.endGame() (package it.unibo.caesena controller)
    \end{itemize}
\end{itemize}

\subsubsection*{Davide Speziali}
In autonomia ho sviluppato:
\begin{itemize}
    \item 
\end{itemize}
In collaborazione ho sviluppato:
\begin{itemize}
    \item 
\end{itemize}

\subsubsection*{Samuele Giancarli}
In autonomia ho sviluppato:
\begin{itemize}
    \item 
\end{itemize}
In collaborazione ho sviluppato:
\begin{itemize}
    \item 
\end{itemize}

\subsection{Note di sviluppo}
\subsubsection*{Mauro Pellonara} 

\subsubsection*{Alessandro Martini}
\begin{itemize}
    \item Utilizzo di Stream: sono presenti all'interno delle classi sviluppati porzioni di codice contenente il costrutto funzionale Stream: \url{https://github.com/MauroPello/OOP22-caesena/blob/2e7e1877928505c25e791d2e1ba6e5ebffffca64/src/main/java/it/unibo/caesena/model/gameset/GameSetImpl.java}
    \item Uso di JSON Parser: utilizzo di un JSON parser per prendere tutte le tile presenti nel gioco (memorizzate nel file config.json) \url{https://github.com/MauroPello/OOP22-caesena/blob/2e7e1877928505c25e791d2e1ba6e5ebffffca64/src/main/java/it/unibo/caesena/controller/ConfigurationLoader.java}
\end{itemize}

\subsubsection*{Davide Speziali}

\subsubsection*{Samuele Giancarli}
