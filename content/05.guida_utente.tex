\section{Guida utente}
La schermata iniziale del gioco presenta la scelta del numero di giocatori, e per ciascuno, dei campi in cui inserirne il nome e il relativo colore. Dopo aver selezionato e impostato a piacimento le informazioni, si dovrà semplicemente cliccare sul pulsante d'avvio della partita.

A questo punto si avrà accesso alla schermata di gioco, che si presenta come in figura. L'elemento principale del gioco è il tabellone che mostrerà al suo centro la tessera iniziale vicino a cui dovranno essere piazzate le altre tessere. In basso a sinistra dello schermo vengono mostrati colore e nome del giocatore corrente oltre al numero dei suoi seguaci rimanenti, mentre alla destra si trovano la classifica dei giocatori, il pulsante "ruota" e l'immagine della tessera corrente.

Premendo il pulsante "ruota" la tessera viene ruotata di un lato in senso orario

a destra dello schermo si presentano i controlli del tabellone che ne gestiscono lo Zoom e permettono al giocatore di spostarsi nella griglia.

premendo "Piazza tessera" questa viene posizionata in caso sia stata selezionata una posizione nel tabellone

premendo "Scarta" verrà scartata la tessera se non piazzabile

Premendo "Piazza seguace" si apre la finestra che permette il piazzamento del seguace

Premendo "Finisci turno" si passa direttamente direttamente alla schermata del giocatore successivo





% Live-Game: Al loro turno, i giocatori pescheranno in maniera casuale una tessera dalla pila e la dovranno aggiungere al tabellone di gioco. Bisognerà posizionare la tessera adiacente a un'altra tessera sul tabellone e si dovrà far coincidere i tipi di terreno della tessera con quelli delle tessere esistenti (cliccando il tasto PlaceTile verrà confermata il posizionamento della tessera). Ad esempio, una tessera strada deve essere posizionata accanto a una tessera strada, una tessera città deve essere posizionata accanto a una tessera città e così via. Un meeple può essere posizionato su una tessera città (per esempio) e segna punti quando la città viene completata. In caso non vi sia possibilità di inserire la tessera sulla mappa, il giocatore potrà far ruotare la propria tessera (cliccando sulla rotellina della freccia, presente affianco alla tessera) finchè non troverà un verso e una posizione in cui inserire la tessera. Dopo aver piazzato la tessera, il giocatore ha la possibilità di posizionarvi sopra una delle sue pedine, chiamate meeples (cliccando il tasto PlaceMeeple) e per terminare il proprio turno, il giocatore dovrà cliccare il tasto endTurn. Il gioco proseguirà finchè non verranno esaurite le tessere.

Alla fine della partita, verrà mostrata la classifica finale nella quale ogni giocatore vedrà il proprio punteggio conclusivo e avrà la possibilità di tornare alla schermata iniziale o di uscire dall'applicativo. 