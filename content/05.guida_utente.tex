\section{Guida utente}
Quando viene fatto partire l'applicativo, gli utenti dovranno inserire il loro nome e il loro colore all'interno del gioco. Ecco come funziona il gioco: 
\begin{itemize}
    \item Schermata iniziale: il gioco si presenta con la scelta del numero di giocatori, e per ciascuno dovranno essere insieriti il nome e il relativo colore per i meeple. Dopo aver selezionato e impostato a piacimento le informazioni, si dovrà semplicemente cliccare sul bottone "start game". I giocatori iniziano la partita posizionando la tessera iniziale al centro del tabellone (già auto-implementata dal nostro gioco).
    \item Live-Game: Al loro turno, i giocatori pescheranno in maniera casuale una tessera dalla pila e la dovranno aggiungere al tabellone di gioco. Bisognerà posizionare la tessera adiacente a un'altra tessera sul tabellone e si dovrà far coincidere i tipi di terreno della tessera con quelli delle tessere esistenti (cliccando il tasto PlaceTile verrà confermata il posizionamento della tessera). Ad esempio, una tessera strada deve essere posizionata accanto a una tessera strada, una tessera città deve essere posizionata accanto a una tessera città e così via. Un meeple può essere posizionato su una tessera città (per esempio) e segna punti quando la città viene completata. In caso non vi sia possibilità di inserire la tessera sulla mappa, il giocatore potrà far ruotare la propria tessera (cliccando sulla rotellina della freccia, presente affianco alla tessera) finchè non troverà un verso e una posizione in cui inserire la tessera. Dopo aver piazzato la tessera, il giocatore ha la possibilità di posizionarvi sopra una delle sue pedine, chiamate meeples (cliccando il tasto PlaceMeeple) e per terminare il proprio turno, il giocatore dovrà cliccare il tasto endTurn. Il gioco proseguirà finchè non verranno esaurite le tessere.
    \item Schermata finale: Alla fine della partita, i giocatori ottengono punti anche per i loro meeples sui campi incompleti. Il giocatore con il maggior numero di punti alla fine della partita è il vincitore. 
\end{itemize}