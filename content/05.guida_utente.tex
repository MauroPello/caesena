\section{Guida utente}
Quando viene fatto partire l'applicativo, gli utenti dovranno inserire il loro nome e il loro colore all'interno del gioco. Ecco come funziona il gioco: 
\begin{itemize}
    \item Impostazione del gioco: il gioco è composto da una mappa e da un set di tessere. La mappa mostra una vasta area di terra, divisa in diverse regioni, tra cui città, strade, campi e monasteri. I giocatori iniziano la partita posizionando la tessera iniziale al centro del tabellone (già auto-implementata dal nostro gioco).
    \item Live-Game: Al loro turno, i giocatori pescano una tessera dalla pila e la aggiungono al tabellone di gioco. Devono posizionare la tessera adiacente a un'altra tessera sul tabellone e devono far coincidere i tipi di terreno della tessera con quelli delle tessere esistenti (cliccando il tasto PlaceTile). Ad esempio, una tessera strada deve essere posizionata accanto a una tessera strada, una tessera città deve essere posizionata accanto a una tessera città e così via. Dopo aver piazzato la tessera, il giocatore ha la possibilità di posizionarvi sopra una delle sue pedine, chiamate meeples (cliccando il tasto PlaceMeeple). Un meeple può essere posizionato su una tessera città (per esempio) e segna punti quando la città viene completata. 
    \item Punteggio: I punti vengono assegnati nel corso del gioco per il completamento di città, strade e monasteri. Quando una città, una strada o un monastero viene completato, il giocatore che ha il maggior numero di meeples su di esso ottiene i punti. Se più giocatori hanno meeples su un elemento completato, ognuno di loro ottiene i punti. Alla fine della partita, i giocatori ottengono punti anche per i loro meeples sui campi completati. Il giocatore con il maggior numero di punti alla fine della partita è il vincitore.
\end{itemize}