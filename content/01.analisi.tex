\section{Analisi dei requisiti}
\subsection{Intervista}
Si vuole realizzare la trasposizione videoludica del gioco da tavolo Carcassonne \cite{Carcassonne} nel quale i giocatori devono posizionare tessere e seguaci per formare strutture e realizzare punti. Chi alla fine della partita avrà totalizzato più punti verrà proclamato vincitore.
\medskip

Ogni giocatore sarà identificato univocamente dal nome, potrà partecipare a più partite diverse contemporeaneamente e avrà un colore e un punteggio distinto per ognuna di esse.
\medskip

Ogni partita dovrà permettere l'uso di più espansioni, ovvero insiemi di diverse tessere, strutture e seguaci. Inoltre, si potrà mettere in pausa una partita e riprenderla in un secondo momento.
\medskip

Le tessere hanno un ordine preciso nella quale devono esse giocate e sono di diversi tipi, ogni tipo è contraddistinto dall'espansione alla quale appartiene e da quante tessere di quel tipo vanno usate in una partita. Inoltre, prima di essere posizionata sul campo ogni tessera può essere ruotata.
\medskip

All'interno di una tessera sono presenti più sezioni, contraddistinte dalla struttura e dall'eventuale seguace su di loro presente. Ognuna di queste dovrà poter essere considerata chiusa, ovvero combaciante con un'altra sezione di una tessera adiacente. È quindi necessario per ogni tessera definire un modo per poterla configurare alla sua creazione.
\medskip

Le strutture sono di diversi tipi e possono contenere dei punti ed essere considerate chiuse. Ogni tipo è contraddistinto dall'espansione alla quale appartiene, da quanti punti una struttura di quel tipo contiene alla creazione e dal numero per cui dividere i punti rimasti a fine partita.
\medskip

I seguaci possono essere piazzati su una sezione di una tessera, sono di diversi tipi ed appartengono ad un giocatore. Ogni tipo è contraddistinto dalla propria forza, dall'espansione alla quale appartiene e da quanti seguaci di quel tipo vanno assegnati ad ogni giocatore.
\medskip

Ulteriormente, in vista della possibile aggiunta futura di una modalità online, bisognerà permettere ai giocatori di scegliere la regione e il server nella quale giocare. È quindi necessario tenere traccia di quante partite sono giocate su uno stesso server e se questo è acceso o spento.

\begin{figure}[ht]
    \centering\includegraphics[scale=0.6]{images/CittàMeeple.png}
    \caption{Esempio di una struttura di tipo Città con sopra un seguace.}
\end{figure}

\subsection{Estrazione dei concetti principali}
\centerline{\begin{tabular}{ |l|p{3in}|c| }
\hline
\textbf{Termine} & \textbf{Breve descrizione} & \textbf{Eventuale sinonimi} \\
\hline
Giocatore & Persona che partecipa a diverse partite, è in grado di ottenere punti dal piazziamento di seguaci & Player, Utente \\
\hline
Partita & Istanza di gioco, termina se tutte le tessere sono state piazzate. Al termine viene decretato il giocatore vincitore & Game\\
\hline
Espansione & Insieme di tessere, strutture e seguaci aggiuntivi & Expansion \\
\hline
Tessera & Elemento di gioco da piazziare a ogni turno della partita in modo tale che combaci con le tessere precedenti & Tassello, tile\\
\hline
Sezione & Parte di una tessera contenente una struttura sulla quale è possibile piazzare un seguace & Section\\
\hline
Struttura & Componente di gioco che permette di ricevere punti in caso di presenza di un seguace & Gameset\\
\hline
Seguace & Componente di gioco posizionabile nelle strutture, alla chiusure di esse il seguace viene restituito al giocatore & Pedina, Meeple\\
\hline
Server & Sistema o piattaforma che ospita le partite online & \\
\hline
Regione & Area geografica o regionale nella quale giocare & Region, area \\
\hline
\end{tabular}}
\medskip

Dopo aver letto e compreso i requisiti, si procede a scrivere un documento che riassuma tutte le idee e, in particolare, evidenzi i concetti principali eliminando eventuali ambiguità.
\medskip

Ogni \textbf{\emph{giocatore}}, identificato univocamente con un nome, deve essere in grado di partecipare a più partite, anche contemporaneamente. Sono presenti informazioni legate al giocatore che valgono solo all'interno di una \textbf{\emph{partita}}, in particolar modo il punteggio e il \textbf{\emph{colore}}, per questo motivo diventa necessario considerare l'esistenza di un \textbf{\emph{giocatore in partita}}.
\medskip

Inoltre per ogni partita bisogna salvarne lo stato di completamento. Ogni partita può fare uso di diverse \textbf{\emph{espansioni}}, che possono contenere nuovi seguaci, nuove tessere e nuove strutture. Inoltre per ogni partita devono essere univocalmente presenti un determinato numero di seguaci, tessere e strutture, che non possono quindi esistere fuori dal contesto di una partita.
\medskip

Per ogni \textbf{\emph{seguace}} bisogna sapere il giocatore proprietario, il tipo di seguace e se è correntemente piazzato all'interno di una sezione.
\medskip

Il \textbf{\emph{tipo di seguace}} esiste al di fuori del contesto della partita e determina, di un dato seguace, l'espansione di cui fa parte, la sua forza e la quantità di seguaci di suddetto tipo che vanno assegnati a ogni giocatore.
\medskip

Di ogni \textbf{\emph{tessera}} è necessario salvare l'ordine all'interno della partita, la posizione sul tabellone e il tipo. Ulteriormente, una tessera è composta anche da diverse \textbf{\emph{sezioni}}. Ognuna di queste contiene una \textbf{\emph{struttura}} ed ha un certo \textbf{\emph{tipo di sezione}} che determina le sezioni circostanti. È anche necessario avere un \textbf{\emph{configuratore di tipo di tile}} che definisce la correlazione di un tipo di tile, il tipo delle sezioni e il tipo delle strutture presenti su di esse.
\medskip

Il \textbf{\emph{tipo di tessera}} esiste al di fuori del contesto della partita e determina, sulla base di una data tessera, l'espansione di cui fa parte e la quantità di tessere di quel tipo che devono essere presenti in una partita.
\medskip

Di ogni struttura bisogna sapere se è chiusa o meno e il punteggio. Le strutture possono essere di diversi tipi e il \textbf{\emph{tipo di struttura}}, che esiste al di fuori del contesto di una partita, indica l'espansione alla quale appartiene, quanti punti una struttura di quel tipo contiene alla creazione e il numero per cui dividere i punti rimasti a fine partita.
\medskip

Le partite sono giocate all'interno di un \textbf{\emph{server}} e per ognuno se ne memorizza lo stato (se è attivo o meno) e il numero massimo di partite giocabili. Per ogni server bisogna anche sapere la \textbf{\emph{regione}} in cui si trova, che è composta dal \textbf{\emph{continente}} e dal \textbf{\emph{punto cardinale}}.
\medskip

Segue per chiarezza quali elementi esistono nel contesto di una partita e quali invece sono universali:

\centerline{\begin{tabular}{ |l|c| }
\hline
\textbf{Elemento} & \textbf{Univoco ad una partita} \\
\hline
Struttuta & Si \\
\hline
Giocatore in partita & Si \\
\hline
Seguace & Si \\
\hline
Settore & Si \\
\hline
Tessera & Si \\
\hline
Giocatore & No \\
\hline
Espansioni & No \\
\hline
Tipo di seguace & No \\
\hline
Tipo di struttura & No \\
\hline
Tipo di tessera & No \\
\hline
Tipo di sezione & No \\
\hline
Server e regione & No \\
\hline
Configuratore di tipo di tile & No \\
\hline
\end{tabular}}
\medskip

Si conclude con una lista delle principali azioni richieste:
\begin{itemize}
    \item Creare una nuova partita
    \item Entrare in una partita in corso
    \item Vedere delle statistiche globali
\end{itemize}



