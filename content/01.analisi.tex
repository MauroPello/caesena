\section{Analisi dei requisiti}
\subsection{Intervista}
Si vuole realizzare la trasposizione videoludica del gioco da tavolo Carcassonne \cite{Carcassonne} nel quale i giocatori devono posizionare tessere e seguaci per formare strutture e realizzare punti. Chi alla fine della partita avrà totalizzato più punti verrà proclamato vincitore.
\medskip

Ogni giocatore sarà identificato univocamente dal nome, potrà partecipare a più partite diverse contemporeaneamente e avrà un colore e un punteggio distinto per ognuna di esse.
\medskip

Ogni partita dovrà permettere l'uso di più espansioni, ovvero insiemi di diverse tessere, strutture e seguaci. Inoltre, si potrà mettere in pausa una partita e riprenderla in un secondo momento.
\medskip

Le tessere hanno un ordine preciso nella quale devono esse giocate e sono di diversi tipi, ogni tipo è contraddistinto dall'espansione alla quale appartiene e da quante tessere di quel tipo vanno usate in una partita. Inoltre, prima di essere posizionata sul campo ogni tessera può essere ruotata.
\medskip

All'interno di una tessera sono presenti più sezioni, contraddistinte dalla struttura e dall'eventuale seguace su di loro presente. Ognuna di queste dovrà poter essere considerata chiusa, ovvero combaciante con un'altra sezione di una tessera adiacente. È quindi necessario per ogni tessera definire un modo per poterla configurare alla sua creazione.
\medskip

Le strutture sono di diversi tipi e possono contenere dei punti ed essere considerate chiuse. Ogni tipo è contraddistinto dall'espansione alla quale appartiene, da quanti punti una struttura di quel tipo contiene alla creazione e dal rapporto per cui dividere i punti rimasti a fine partita.
\medskip

I seguaci possono essere piazzati su una sezione di una tessera, sono di diversi tipi ed appartengono ad un giocatore. Ogni tipo è contraddistinto dalla propria forza, dall'espansione alla quale appartiene e da quanti seguaci di quel tipo vanno assegnati ad ogni giocatore.
\medskip

Ulteriormente, in vista della possibile aggiunta futura di una modalità online, bisognerà permettere ai giocatori di scegliere la regione e il server nella quale giocare. È quindi necessario tenere traccia di quante partite sono giocate su uno stesso server e se questo è acceso o spento.

\begin{figure}[ht]
    \centering\includegraphics[scale=0.6]{images/CittàMeeple.png}
    \caption{Esempio di una struttura di tipo Città con sopra un seguace.}
\end{figure}

\subsection{Estrazione dei concetti principali}
(rilevamento delle ambiguità con relative correzioni proposte e definizione delle specifiche in linguaggio naturale)

\centerline{\begin{tabular}{ |l|c|c| }
    \hline
    \textbf{Termine} & \textbf{Breve descrizione} & \textbf{Eventuale sinonimi} \\
    \textbf{Termine} & \textbf{Breve descrizione} & \textbf{Possibili sinonimi} \\
    \hline
    Giocatore & Persona che partecipa a diverse partite, è in grado di ottenere punti dal piazziamento di seguaci & Player, Utente \\
    \hline
    Partita & Istanza di gioco, termina se tutte le tessere sono state piazzate. Al termine viene decretato il giocatore vincitore & Game\\
    \hline
    Espansione & Insieme di tessere, strutture e seguaci aggiuntivi & Expansion \\
    \hline
    Tessera & Elemento di gioco da piazziare a ogni turno della partita in modo tale che combaci con le tessere precedenti & Tassello, tile\\
    \hline
    Sezione & Parte di una tessera contenente strutture sulla quale è possibile piazzare seguaci & Section\\
    \hline
    Struttura & Componente di gioco che permette di ricevere punti in caso di presenza di seguace & Gameset\\
    \hline
    Seguace & Componente di gioco posizionabile nelle strutture, alla chiusure di esse il seguace viene ridato al giocatore. Quest'ultimo riceve punteggio a seconda della struttura & Pedina, Meeple\\
    \hline
    Server & Sistema o piattaforma che ospita le partite online & \\
    \hline
    Regione & Area geografica o regionale nella quale giocare & Region, area \\


    \end{tabular}}

    Dopo aver letto e compreso i requisiti, si procede a scrivere un documento che riassuma tutte le idee e, in particolare, evidenzi i concetti principali eliminando eventuali le ambiguità.
    Ogni \textbf{\emph{giocatore}}, identificato univocamente con un nome, deve essere in grado di partecipare a più partite, anche contemporaneamente. Sono presenti informazioni legate al giocatore che valgono solo all'interno di una \textbf{\emph{partita}}, in particolar modo il punteggio e il colore identificativo, per questo motivo diventa necessario considerare l'esistenza di un \textbf{\emph{giocatore in partita}}. Inoltre per ogni partita bisogna salvare lo stato di completamento. Ogni partita può fare uso di diverse \textbf{\emph{espansioni}}, che definisco modifiche al gioco base e possono quindi contenere nuovi seguaci, nuove tessere e nuove strutture. Inoltre per ogni partita devono essere univocalmente presenti un determinato numero di seguaci, tessere e strutture, che non possono quindi esistere fuori dal contesto di una partita. Per ogni \textbf{\emph{seguace}} bisogna sapere il proprio giocatore, il tipo di seguace e se è correntemente piazzato all'interno di un settore. Il \textbf{\emph{tipo di seguace}} esiste al di fuori del contesto della partita e determina, di un dato seguace, l'espansione di cui fa parte, la sua forza e la quantità di seguaci di suddetto tipo che vengono assegnati a ogni giocatore. Di ogni \textbf{\emph{tessera}} è necessario salvare il loro ordine all'interno della partita e il loro tipo. Il \textbf{\emph{tipo di tessera}} esiste al di fuori del contesto della partita e determina, sulla base di una determinata tessera, l'espansione di cui fa parte e la quantità di tessere di quel tipo che devono essere presenti in una partita. Ogni tessera è composta da diversi \textbf{\emph{settori}}, che devono contenere una \textbf{\emph{struttura}}. Ogni settore è ha un tipo e il \textbf{\emph{tipo di settore}} determina i settori circostanti. È anche necessario avere un \textbf{\emph{configuratore di tipo di tile}} che definisce la correlazione di un tipo di tile, il tipo dei sui settori e il tipo delle strutture presenti nei settori. Di ogni struttura bisogna sapere se è chiusa o meno e il loro punteggio. Le strutture possono essere di diversi tipi e il \textbf{\emph{tipo di struttura}}, che esiste al di fuori del contesto di una partita, indica l'espansione alla quale appartiene, da quanti punti una struttura di quel tipo contiene alla creazione e dal rapporto per cui dividere i punti rimasti a fine partita. Le partite sono giocate all'interno di un \textbf{\emph{server}} e per ognuno si memorizza il loro stato(se attivo o meno), il numero massimo di partite giocabili. Per ogni server bisogna anche sapere la \textbf{\emph{regione}} in cui si trova, che è composta dal \textbf{\emph{continente}} e dal \textbf{\emph{punto cardinale}}.

    Segue per chiarezza quali elementi esistono nel contesto di una partita e quali invece sono universali:
    \centerline{\begin{tabular}{ |l|c| }
        \hline
        \textbf{Elemento} & \textbf{Univoco ad una partita} \\
        \hline
        Giocatore & No \\
        \hline
        Giocatore in partita & Si \\
        \hline
        Espansioni & No \\
        \hline
        Seguace & Si \\
        \hline
        Tipo di seguace & No \\
        \hline
        Tessera & Si \\
        \hline
        Tipo di tessera & No \\
        \hline
        Struttuta & Si \\
        \hline
        Tipo di struttura & No \\
        \hline
        Settore & Si \\
        \hline
        Tipo di settore & No \\
        \hline
        Configuratore di tipo di tile & No \\
        \hline
        Server e regione & No \\
        \hline
        \end{tabular}}

    Si conclude con una lista delle principali azioni richieste:
    \begin{enumerate}
        \item Creare una nuova partita
        \item Entrare in una partita in corso
        \item Vedere delle statistiche globali
      \end{enumerate}



