\section{Analisi}
\subsection{Requisiti}
Il progetto si pone l'obiettivo di ricreare il gioco da tavolo "Carcassone" 1 ispirandosi anche alla sua controparte videoludica 2. 

\subsubsection*{Requisiti funzionali}
\begin{itemize}
\item Calcolo del punteggio progressivo
\item Calcolo del punteggio al termine del gioco
\item Menù principale, con selezione numero giocatori
\item Gestione turno del giocatore
\item Gestione posizionamento pedine
\item Calcolo delle posizioni valide per le tessere
\item Gestione della creazione o ampliamento di prati, città e strade
\end{itemize}

\subsubsection*{Requisiti non funzionali}
\begin{itemize}
\item Menù di pausa e salvataggio sessione
\item Possibilità di visualizzare le tessere rimanenti
\item Scelta del colore dell’avatar
\item Gestione dell’audio
\item Aggiunta di espansioni del gioco (nuove tessere e regole modificate)
\item Sviluppo di una semplice AI per permettere una modalità singleplayer
\end{itemize}

\subsection{Analisi e modello del dominio}
Caesena permette ai giocatori di poter piazzare delle pedine figurate (Tile) e dei pupazzetti (Meeple); alla chiusura dell'immagine raffigurata composta, se presenti meeple all'interno dell'immagine, il giocatore relativo al meeple prenderà tanti punti tanti quanti il valore associato alla figura chiusa e i meeple presenti su quell'immagine verrano ripristinati al relativo proprietario.
Ogni tile può rappresentare più di una tipologia di figura differente, ogni singola tipologia di figura viene chiamata GameSet. Il giocatore, per poter piazzare la tile, dovrà far combaciare tutti i gameset della tile da piazzare con almeno un gameset di tile precedentemente posizionate.
In caso la tile mostrata inizialmente non permette di essere posizionata, il giocatore potrà ruotare la tile ogni volta di novanta gradi. Le difficoltà presenti all'interno del progetto sono rivolte sopratutto rivolte alla parte grafica, infatti la prima versione software verrà fornita con un'interfaccia minimale.
