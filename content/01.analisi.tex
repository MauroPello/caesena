\section{Analisi}
\subsection{Requisiti}
Il progetto si pone l'obiettivo di ricreare il gioco da tavolo "Carcassonne", con la possibilità di poter far giocare da 2 a 6 giocatori e la scelta individuale del proprio colore. Durante il gioco il gioco, il giocatore potrà piazzare tessere (obbligatoriamente) e pupazzetti (a discrezione) al fine di poter prendere più punti possibili. 

\subsection*{Requisiti funzionali}
\subsubsection*{Requisiti funzionali obbligatori}
\begin{itemize}
\item Calcolo del punteggio progressivo
\end{itemize}
\subsubsection*{Requisiti funzionali non obbligatori}
\begin{itemize}
\item Calcolo del punteggio progressivo
\end{itemize}

\subsubsection*{Requisiti non funzionali}
\begin{itemize}
\item Menù di pausa e salvataggio sessione
\end{itemize}

\subsection{Analisi e modello del dominio}
L'analisi e il modello di dominio di Carcassonne possono essere suddivisi in diversi componenti chiave, tra cui il tabellone di gioco, le tessere, i meeples e il sistema di punteggio.
\begin{itemize}
    \item Tabellone di gioco (Mappa): Il tabellone di Carcassonne rappresenta la regione di Carcassonne ed è composto da una serie di tessere che vengono posizionate sul tabellone durante il gioco. Il tabellone è suddiviso in diverse regioni, tra cui città, strade, campi e monasteri. Ognuna di queste regioni ha le proprie regole e il proprio sistema di punteggio, che aggiungono profondità e complessità al gioco.
    \item Le tessere (Tiles): Le tessere di Carcassonne rappresentano diversi tipi di terreno, tra cui città, strade, campi e monasteri. Queste tessere vengono estratte casualmente da una pila e devono essere posizionate sul tabellone di gioco in modo che corrispondano al tipo di terreno delle tessere circostanti. Ogni tessera ha le sue regole e il suo sistema di punteggio, che i giocatori devono seguire per massimizzare i loro punti.
    \item Meeples: I meeple sono i pezzi di gioco che i giocatori usano per controllare le diverse regioni del tabellone. 
    \item Sistema di punteggio: Il sistema di punteggio di Carcassonne si basa sul completamento di diverse regioni sul tabellone di gioco. Quando una città, una strada o un monastero vengono completati, il giocatore che ha il maggior numero di meeples su di essi ottiene i punti. Se più giocatori hanno meeples su un elemento completato, ognuno di loro ottiene i punti. Alla fine della partita, i giocatori ottengono punti anche per i loro contadini sui campi completati. Il giocatore con il maggior numero di punti alla fine del gioco è il vincitore.
\end{itemize}
Le difficoltà che il progetto mantiene sono sopratutto presenti nella parte grafica, non ancora ottimizzata a causa dell'utilizzo del framework grafico Java Swing.
