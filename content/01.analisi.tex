\section{Analisi}
\subsection{Requisiti}
Il progetto si pone l'obiettivo di ricreare il gioco da tavolo "Carcassone" 1 ispirandosi anche alla sua controparte videoludica 2.

\section{Funzionalità obbligatorie}
\begin{itemize}
\item Calcolo del punteggio progressivo
\item Calcolo del punteggio al termine del gioco
\item Menù principale, con selezione numero giocatori
\item Gestione turno del giocatore
\item Gestione posizionamento pedine
\item Calcolo delle posizioni valide per le tessere
\item Gestione della creazione o ampliamento di prati, città e strade
\end{itemize}

\section{Funzionalità non obbligatorie}
\begin{itemize}
\item Menù di pausa e salvataggio sessione
\item Possibilità di visualizzare le tessere rimanenti
\item Scelta del colore dell’avatar
\item Gestione dell’audio
\item Aggiunta di espansioni del gioco (nuove tessere e regole modificate)
\item Sviluppo di una semplice AI per permettere una modalità singleplayer
\item \end{itemize}

\section{"Challenge" principali:}
\begin{itemize}
\item Gestione delle tessere, in particolar modo come vengono gestite le zone (prateria, città, strada) all’unione
\item Corretto utilizzo del pattern MVC
\item Gestire la GUI considerando che il campo di gioco è potenzialmente infinito
\item Mantenimento del codice sorgente attraverso fork e pull-request
\end{itemize}

\subsection{Analisi e modello del dominio}
prova