\section{Analisi dei requisiti}
\subsection{Intervista}
Si vuole realizzare la trasposizione videoludica del gioco da tavolo Carcassonne \cite{Carcassonne} nel quale i giocatori devono posizionare tessere e seguaci per formare strutture e realizzare punti. Chi alla fine della partita avrà totalizzato più punti verrà proclamato vincitore.
\medskip

Ogni giocatore sarà identificato univocamente dal nome, potrà partecipare a più partite diverse contemporeaneamente e avrà un colore e un punteggio distinto per ognuna di esse.
\medskip

Ogni partita dovrà permettere l'uso di più espansioni, ovvero insiemi di diverse tessere, strutture e seguaci. Inoltre, si potrà mettere in pausa una partita e riprenderla in un secondo momento.
\medskip

Le tessere hanno un ordine preciso nella quale devono esse giocate e sono di diversi tipi, ogni tipo è contraddistinto dall'espansione alla quale appartiene e da quante tessere di quel tipo vanno usate in una partita. Inoltre, prima di essere posizionata sul campo ogni tessera può essere ruotata.
\medskip

All'interno di una tessera sono presenti più sezioni, contraddistinte dalla struttura e dall'eventuale seguace su di loro presente. Ognuna di queste dovrà poter essere considerata chiusa, ovvero combaciante con un'altra sezione di una tessera adiacente. È quindi necessario per ogni tessera definire un modo per poterla configurare alla sua creazione.
\medskip

Le strutture sono di diversi tipi e possono contenere dei punti ed essere considerate chiuse. Ogni tipo è contraddistinto dall'espansione alla quale appartiene, da quanti punti una struttura di quel tipo contiene alla creazione e dal rapporto per cui dividere i punti rimasti a fine partita.
\medskip

I seguaci possono essere piazzati su una sezione di una tessera, sono di diversi tipi ed appartengono ad un giocatore. Ogni tipo è contraddistinto dalla propria forza, dall'espansione alla quale appartiene e da quanti seguaci di quel tipo vanno assegnati ad ogni giocatore.
\medskip

Ulteriormente, in vista della possibile aggiunta futura di una modalità online, bisognerà permettere ai giocatori di scegliere la regione e il server nella quale giocare. È quindi necessario tenere traccia di quante partite sono giocate su uno stesso server e se questo è acceso o spento.

\begin{figure}[ht]
    \centering\includegraphics[scale=1]{images/CittàMeeple.png}
    \caption{Esempio di una struttura di tipo Città con sopra un seguace.}
\end{figure}

\subsection{Estrazione dei concetti principali}
(rilevamento delle ambiguità con relative correzioni proposte e definizione delle specifiche in linguaggio naturale)
