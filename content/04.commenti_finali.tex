\section{Commenti finali}
\subsection{Autovalutazione e lavori futuri}
\subsubsection*{Mauro Pellonara}
Sono molto soddisfatto di questo progetto in quanto mi ha permesso di allenare le mie capacità di progettazione, astrazione e soprattutto programmazione. È stato molto motivante lavorare con gli altri colleghi facenti parte del gruppo, in quanto ci siamo spronati continuamente a cercare di scrivere codice sempre più pulito e corretto. Probabilmente la fase che ho apprezzato di più è stata quella di progettazione, iniziata con una partita al gioco da tavolo originale e conclusa spendendo numerose ore alla lavagna abbozzando il progetto. Devo però anche dire che in questa fase ci saremmo dovuti concentrare di più sul progettare la parte grafica, che invece è stata fatta inizialmente in modo confusionario e poi aggiustata e corretta. All'interno del gruppo, oltre a programmare, ho anche avuto il compito di organizzare e coinvolgere tutti i miei colleghi, promuovendo obiettivi parziali ogni settimana e cercando di distribuire al meglio il lavoro da fare. Avrei decisamente apprezzato una partecipazione più attiva e puntuale da parte di tutti i miei colleghi, cosa che invece non c'è stata fin dall'inizio e in alcuni casi nemmeno alla fine.

\subsubsection*{Alessandro Martini}
Sono soddisfatto del mio lavoro all'interno del progetto, mi ritengo molto fortunato e orgoglioso per aver lavorato con colleghi che ritengo molto validi e collaborativi a questo progetto.
Lo scambio di idee e il supporto reciproco sono stati cardine durante l'intero lavoro di progetto. Questo è stato il mio primo vero grande progetto scritto, e ho imparato l'utilità della comunicazione tra colleghi, ma sopratutto ho compreso l'utilità di scrivere codice pulito, riusabile all'interno di progetti, sopratutto per una corretta interpretazione futura di esso. Ritengo la parte grafica il tallone d'achille di questo progetto che, una volta terminato, cerceremo di migliorare.
\subsubsection*{Davide Speziali}

\subsubsection*{Samuele Giancarli}
Ho trovato estremamente gratificante questo lavoro in quanto mi ha mostrato di avere già le capacità e gli strumenti per poter sviluppare un'applicazione in java completamente funzionante, mettendo inoltre in pratica l'utilizzo di pattern e stili di programmazione appresi a lezione. Essendo l'elemento del team con la maggior consapevolezza del gioco originale ho avuto un importante ruolo nella sua progettazione iniziale, che mi ha stimolato in quanto ha necessitato di un grande lavoro di astrazione delle meccaniche di gioco da parte di tutto il gruppo. Ho trovato il gruppo di facile  decisamente supportivo e valido. Probabilmente la possibilità di aggiungere altre tessere al gioco con annesse nuove meccaniche mi porterà ad aggiornare il programma quando sarà possibile, magari per poterlo anche utilizzare come portfolio personale. Mi sento di muovere una piccola critica sulla distribuzione dei compiti in fase di programmazione che sarebbe potuta essere più omogenea.

%discutere del proprio ruolo nel gruppo?